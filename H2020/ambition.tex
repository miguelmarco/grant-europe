\subsection{Ambition}



\eucommentary{1-2 pages}

\eucommentary{-- Describe the advance your proposal would provide beyond the
state-of-the-art, and the extent the proposed work is ambitious. Your answer
could refer to the ground-breaking nature of the objectives, concepts
involved, issues and problems to be addressed, and approaches and methods to be used.\\
-- Describe the innovation potential which the proposal represents. Where relevant, refer to
products and services already available, e.g. in existing
e-Infrastructures.}

For most pure mathematicians using computational tools in their
research, the state of the art in 2014 is still a collection of
programs each of which must be installed individually on their
desktop or laptop computer, respecting a complicated graph of
dependencies. Alternatively software may be installed on a
departmental server or cluster and used via text-based remote
login. The software performs computations (using excellent
implementations of extremely sophisticated algorithms) with inputs and
outputs usually in a bespoke text-based format. The results of
computations are incorporated into publications by cut-and-paste and
collaboration is through exchange of programs and data by email,
shared general-purpose file servers or, rarely, a service such as
GitHub. Multiple computations involved in producing a mathematical
result must be managed by editing, naming and filing multiple scripts
or programs, and there is no automatic support for rerunning
computations to check for human or algorithmic error. 

\TODO{Work out best order for these topics}

There are commercial ``symbolic computation systems'' such as
Mathematica \textsuperscript{\textregistered}. or Maple \textsuperscript{\textregistered}. which offer somewhat more modern frameworks, but
they lack the specialised algorithms for research work in fields such
as algebra, number theory or algebraic geometry and are not
well-suited to support them.

The need for a more modern, more productive and less error-prone
environment for this kind of mathematical research computing is widely
acknowledged, but the separate groups developing the systems have
individually, neither the time nor the expertise to develop it. There
have been a number of interesting projects which have explored
different aspects of what is needed: \TODO{SCIEnce, SageMathCloud,
  MathOverflow, Polymath, SAGE itself, SAGE notebooks, HPCGAP,
  recomputation} and we will build on the experiences, and where
useful the software, of all of these.


Our ambitious plan in this project is to learn from, and leapfrog,
these piecemeal developments and provide a toolkit of software and
interfaces, which supports the whole mathematical research process in
a way which is \textbf{modern}, \textbf{seamless},
\textbf{collaborative}, mathematically \textbf{rigourous} and
\textbf{adaptable} to the diverse needs of different mathematical
research areas and of different mathematicians and collaborations.

\TODO{Explain all the highlighted words in the para above}

\TODO{Some examples here -- what will we deliver to individuals; small ``single-problem''
  collaborations; longer-lived data- or algorithm- centred teams;
  massive ``flagshsip'' projects. Think ``User Stories''}

\subsubsection{Challenges specific to  mathematics}

Mathematical research, especially pure mathematics presents some
unique challenges to the realisation of this ambition. 

\TODO{Evidence this in more detail, clean up the language}

\begin{itemize}
\item Community mainly made of individuals or \textit{very} small
  groups (1 PI + a few students). Few formal or structured research
  groups such as you might find in an equipment-intensive
  science. Large scale collaborations happen (CoFSG, Polymath),
  but still driven by individuals, not formal structures or money.
\item Much top quality research has little or no formal research
  funding. So computational resources are limited to what is available
  anyway -- personal laptops, departmental clusters...
\item Many mathematical computations are highly irregular and
  complex. Traditional HPC paradigms coming from simulation and linear
  algebra do not apply.
\item Mathematical notations have been refined over many centuries for
  use by humans with pen, paper and blackboard. Even such simple
  problems as selecting a sub-expression are hard to handle well on a
  computer. For instance $a+c$ is naturally seen as a subexpression of
  $a+b+c$ by a human.
\item The mathematical correctness of widely used algorithms hinges on
  quite complex chains of reasoning. Subtle coding errors may easily
  produce plausible, but wrong, answers.

\item Mathematical data different in several ways from typical
  scientific data
  \begin{itemize}
  \item More often than not data is the result of a computation (and
    not a measurement of the real world). The role of databases is thus primarily
    to store results for later reuse (persistent caching), and enable
    searches. Because of this, many issues (semantic, ontologies,
    reproducibility, ...) are to be treated upstream at the level of
    software rather than data.
  \item extreme reification in mathematics makes classical ontologies
    techniques/RDF impractical \TODO{Someone explain this}
  \item interlinking very high
  \item several alternate and defining description of same objects
  \end{itemize}
\end{itemize}
\subsubsection{Challenges of a community built around multiple
  existing software projects}

Another source of unique challenges for this project is the need to
interact with several large and diverse ecosystems of software
developers. For instance the \GAP\ package development community, the
SAGE development community, the wider Python community, the developers
of key open-source libraries on which we rely and so on.

These communities exist in a delicate balance between collaboration
and competition. For instance the SCIEnce project and SAGE were
simultaneously exploring two different approaches to linking
open-source mathematical software. Many technical developments (better
IO handling in \GAP, for instance) could usefully be shared, and at
the end of the day we all want to do better mathematics, but a certain
degree of competition is both natural and healthy.

In this project we need to build a sustainable ``meta-ecosystem'' in
which systems may compete to have the best designs or algorithms, but
all agree to cooperate on interfaces, bug reporting, testing, etc. to
keep the final user experience seamless and reliable.


Promoting collaboration over competition between communities.
%%% Local Variables:
%%% mode: latex
%%% TeX-master: "proposal"
%%% End:

%  LocalWords:  eucommentary textsuperscript textregistered textsuperscript specialised
%  LocalWords:  textregistered recomputation textbf textbf rigourous centred flagshsip
%  LocalWords:  subsubsection realisation textit
